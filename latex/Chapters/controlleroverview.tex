%Overview of existing technology (hardware)

\chapter{Overview of Existing Technology}
\section{PLC Hardware Controller Implementations}
%link to automotive statement: http://www.amci.com/tutorials/tutorials-what-is-programmable-logic-controller.asp
Programmable Logic Controllers (PLC) have been around for over 30 years and as such there have been many iterations and designs. The original Programmable Logic Controllers came into being to fill the need of automotive manufacturers replacing traditional relays with digital control. These relays were hooked up to power rails and inputs and allowed for basic mechanical logic to be performed. Due to their mechanical nature relays wore out over time causing the logic programmed using them to fail. In addition because hundreds to thousands could be used in a cabinet were also difficult to isolate the worn out part. Relays also proved to be inflexible when a small change was required to be added to the program, the entire plant was required to be taken offline in order to make the change. Halting large production plants is often extremely costly and thus eventually the relays were migrated out in favor of microcontrollers that can be reprogrammed on the fly. To this day modern PLC's still use graphical analogies of circuits and relays in order to construct their programmable logic. This visual language is now referred to as ladder logic due to the the finished program having a similar structure to a conventional ladder. 

\begin{figure}[htp]
    \centering
    \includegraphics[width=\imgmedphoto]{./images/c01_mitsubishiplc.jpg}
    \caption{Mitsubishi PLC All In One Unit \cite{img_c01_MitsubishiPlc}}
    \label{img:mitsubishiplc}
\end{figure}

Mitsubishi Automation\ref{img:mitsubishiplc}, Siemens, and Omron are just a few of the big producers of industry standard PLC's although the shape and form factor differ between manufacturers differ PLC's always consist of 3 distinct parts.  The input module, the main controller unit and the output module (please refer to figure \ref{plcrender_1}). This separation exists due to varying requirements for analog inputs and different output level requirements in order to drive heavy machinery. I/O modules may consist of thermo sensors, ambient light sensors, resistive sensors, or a direct connection the the external circuitry. Likewise the output module may also be composed of both analog or digital output pins.

\begin{figure}[htp]
    \centering
    \includegraphics[width=\imgmedphoto]{./images/c02_plcdev.jpg}
    \includegraphics[width=\imgmedphoto]{./images/c04_plcdev.jpg}
    \caption{3D Diagram of A Modular PLC \cite{img_c02_PlcDev,img_c04_PlcDev}}
    \label{img:plcrender_1}
\end{figure}

%source http://www.sea.siemens.com/step/templates/lesson.mason?plcs:2:3:1
Programs are executed from the main PLC control unit. An iteration of execution is refered to as a scan. A scan is broken up into 4 phases: Self-Test, Input scan, Logic solve / scan, and Output scan. Figure \ref{fig:plcexecution} shows the order in which these steps are executed. Each step contains various jobs, in more detail they are:

\pagebreak{4}

\begin{figure}[htp]
    \centering
    \includegraphics[width=\imgmedphoto]{./images/plcexecution.pdf}
    \caption{PLC Execution Loop}
    \label{fig:plcexecution}
\end{figure}

%todo: draw a flow chart of the following stages
\begin{itemize}
	\item\textbf{Self-Test:} All PLC's contain self diagnostic routines, this includes communication checks between the main control unit and the I/O modules. If a fault is found it is handled here before any of the execution is allowed to proceed.
	\item\textbf{Input Scan:} All inputs both from the input modules and from the internal memory are scaned. This is done in a single step to make sure that all future calculations for the currently executing scan has consistent data. You may note that updates are not read until the next input scan.
	\item\textbf{Logic Solve / Scan:} Calculations and computations from the user programs are computed in this step if values are to be stored back into internal registers they are now put into temporary registers. Similarily if external output is required it is written to a temporary internal register that will hold the output until the output phase is executed.
	\item\textbf{Output Scan:} Internal temporary registers are written to their destination registers in one step. External outputs take on the values held by the registers that stored data for the output modules all outputs also take place in one step.
\end{itemize}

Each of these phases semantically can be assumed to execute cocurrently thus, the order of individual instructions in each phase is of no consequence. This closely follows the Ladder Logic (please see section \ref{section:ladderlogic}) in that the entire program executes cocurrently on multiple rungs. Internally however PLC's do have a sequential deterministic order in which instructions are executed, this is a side effect of using microcontrollers in the main control unit. In this way it can be said that each phase is considered concurrent if we consider a cycle time greater than the time it takes to execute the last instruction. If an external device takes samples at this rate it is irrelevent from its perspective that the processing is not actually happening cocurrently to the observing device the two behaviors are equivalent.

\begin{figure}[htp]
    \centering
    \includegraphics[width=\imgmedphoto]{./images/c03_plcdev.jpg}
    \includegraphics[width=\imgmedphoto]{./images/c05_plcdev.jpg}
    \caption{3D Diagram of A Modular PLC With One Module Being Inserted \cite{img_c03_PlcDev,img_c05_PlcDev}}
    \label{img:plcrender_2}
\end{figure}

The input and output modules generally connect to the main module via serial links however some companies also include network commuication over standard shielded ethernet\cite{rockwell_io,rockwell_tech_pub}. Generally serial communication is used more often when the input and output modules are at a close distance to the controller unit (see figure \ref{img:plcrender_2}) such as in a modular PLC design. The network interface on the other hand is used when the input or output module needs to be located far away from the main controller unit\cite{rockwell_tech_pub} as is often the case in automated production facilities. Output modules are generally relay driven and the driving current is provided by a transistor connected to logic pins of the main controller. This is done to isolate the internal circuitry from the demands of driving heavy machinery \cite{plcapp}. Alternatively some circuits employ an opto-isolator circuitry to achieve the same effect the trade off being less current under load but faster switching and better service life \cite{plcapp}. Analog outputs are obtained by passing a binary value through a DAC with some reference voltage the output voltage.

%source http://www.toolingu.com/class-450240-plc-inputs-and-outputs.html
All input and output modules contain some common  features which allow for modularity. Each I/O module is assigned a unique address so that the controller unit running the PLC program can access it. Each controller also has what is referred to a backpane which contains the necessary connectors to connect to the bus in which the CPU can access. Most I/O modules have multiple channel where each channel is either a single ended wire or a differential pairs. Differential can be commonly seen in analog input modules where it is preferable to compare a signal with the sensor's reference ground in order to avoid problems with the ground from the PLC not matching up with the ground form the sensor. Such a mismatch would produce a floating ground and would induce a perminent error on the input channel. Most modules are designed so that the external input and output enviroments are completely isolated form the internal circuitry. As such, the internal half is connected to common ground via the serial bus while the external half is usually connected to the powersupply via a ground screw (usually marked "common"). In analog input modules conversion circuitry exist to convert the sensed input values into quantized values to be sent over the serial bus. The same is true for output modules but in reverse. In digital inputs conversion is not needed however another circuitry still allows for isolation of the internal bus from the external input, and where applicable the module itself may perform noise correction before the actual input enters the PLC itself. In addition to individual modules some PLC manufacturers opted to have the I/O modules embedded into the main CPU unit. These are generally more commonly found in the micro sized PLC's and are generally less expensive then their expandable counterparts. Although externally they are not expandable internally the configuration of the input modules are actually identical.
%TODO: add here

%source ISBN 0-13-025603-X
Due to the varying requirements of manufacturing over the years a wide variety of I/O modules now exist to fill in every need a short list of common I/O modules is listed below:
\begin{itemize}
	\item \textbf{Analog Input Module \cite{slc500}:} Reads in analog voltages converts them into digital values and sends the information to the control unit.
	\item \textbf{Analog Output Module \cite{slc500}:} Takes in digital values and produces an apporpreate analog voltage level.
	\item \textbf{ASCII Input Module \cite{slc500}:} These modules take in character information usually via serial links (RS-232) and convert them into a form the PLC controllers can understand\cite{slc500}.
	\item \textbf{ASCII Output Module \cite{slc500}:} Takes PLC information converts it to an character representation and sends it out via RS-232.
	\item \textbf{Barrel-Temperature Module \cite{slc500}:} This module monitors four zones of an autotuned PID heating or cooling unit for temperature control. Molding machines and extruder employ these modules for controlling barrel temperature while injecting material\cite{slc500}.
	\item \textbf{BCD Input Module \cite{slc500}:} This module reads in inputs from devices that output binary-coded decimal (a method of representation where each decimal digit is represented by four bits). An example of such a device is a thumb wheel\cite{slc500}.
	\item \textbf{BCD Output Module \cite{slc500}:} Essentially does the reverse of the input module usually used for compatability sake when devices expect data in BCD format.
	\item \textbf{*****************Currently working on this section there's about 6 more I'd like to add once I get my references sorted*******}
	
\end{itemize}

Programmable logic controllers can classified into three categories: Integrated as seen in figure \ref{img:mitsubishiplc}, Modular as seen in figure \ref{img:plcrender_1}, and Large Scale automation. The integrated category includes small one board solutions generally better suited for low power or embedded applications. The modular category consists of PLC's that have a rack that houses the power supply, and several modular slots for both the microcontroller and the input output modules.


%% Additional notes

