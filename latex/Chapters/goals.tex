%goals of this project
\section{Goals}

This project aims to improve on current industrial programmable logic controllers by introducing a more natural graphical programming method. In addition we will evaluate how to create a cost effective alternative using off the shelf parts to construct our own hardware PLC. In the process we aim to produce a final prototype that will have the same basic feature set of modern PLC's. This includes a main controller unit, with input and output capabilities, and a prototype of a basic IDE that will work in our new visual language. In this project we propose state machine diagrams as the method of choice. As state machines can be used to represent all current programming languages. It is also important for this project to understand the deficiencies of Ladder Logic (the current method). In addition we will evaluate the original use of PLC's and if the old methodologies are still applicable to their modern application. This analysis will provide further understanding on how the original programming methods have been outpaced by more recent technology.

Due to the scope of this project we must deliver several components in order to achieve an acceptable proof of concept. The components and their sub-goals are as follows:

We must deliver a main controller unit with an on-board embedded OS. This main controller will serve as our CPU and will include the input and output units in addition to running our program. The main controller must be able to execute adequately fast in order to compete against commercial PLC's. Since the user is not concerned with the execution speed of each instruction we measure fast by the time it takes for an input to trigger an output. We hope to utilize cheaper hardware but still equivalent speeds by allowing the user to specify instructions more closer to the actual chip supported instructions. This reduces the actual number of pseudo instructions required to perform a task. Our hardware will aim to eliminate the concept of a scan (see section \ref{fig:plcexecution}) and in doing so we hope to make better use of available hardware.

We must also design and formalize a visual programming language to be used as a replacement for Ladder Logic. This language will require precise definitions on how to interpret diagram elements. The language should also be designed to be easily understandable and where possible efficient. It should also be advanced enough to construct basic control programs. To practically utilize this language we will deliver a proof of concept IDE that will allow the user to enter programs using the visual programming language instead of Ladder Logic. This new IDE should work like a flow chart drawing program allowing the user to add and remove logic blocks at will. We aim to make this interface as intuitive as possible and also provide a fast efficient translation into machine code. In addition a simulator is also required to help developers visualize a program's execution. The simulator should contain basic step features to allow the programmer to step through each transition, and allow the developer to examine the contents of the variables in memory. The main goal of the IDE will be to provide an easy to understand visual environment where the user may enter their program and to minimize error where possible.

The final layer that will be added is a software to hardware compiler that will take user generated programs from the IDE and produce actual machine code. This will be achieved in two parts. First the IDE will compile the diagram into intermediate code. This intermediate code will be code that will stay hardware neutral and will resemble C. The intermediate code will contain many calls to functions that will be supported on the chip. These functions will be part of our hardware framework that will take the calls placed by intermediate language and translate them into hardware specific calls. From this design choice hardware specific code will stay on the hardware portion of this project and compiled software will stay hardware independent with specific support implemented by the hardware framework.

By delivering an initial proof of concept software and hardware system this project would allow for further development for modernizing programmable logic controllers. In addition it may also serve as a practical example of visual programming languages that are used to program embedded devices.